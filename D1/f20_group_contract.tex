\documentclass[12pt]{article}

\fontfamily{lmss}
\usepackage{fullpage}
\usepackage{amsmath}
\usepackage{amsthm}
\usepackage{url}
\usepackage{multicol}
\usepackage{enumerate}
\usepackage{graphicx}
\usepackage{color}
\usepackage{hyperref}
\hypersetup{
    colorlinks=true,
    linkcolor=blue,
    filecolor=magenta,      
    urlcolor=blue,
}

\usepackage{geometry}
\geometry{
  top=1in,            % <-- you want to adjust this
  bottom=1in,
  left=1in,
  right=1in,
  headheight=3ex,       % <-- and this
  headsep=4ex,          % <-- and this
}

\usepackage{fancyhdr}
\pagestyle{fancy}
\fancyhf{}
\renewcommand{\footrulewidth}{0.4pt}
\lhead{CS 486/686}
\chead{Fall 2020}
\rhead{Project}
\rfoot{Page \thepage}
\cfoot{v1.0}
\lfoot{\copyright Alice Gao 2020}

\setlength{\parskip}{\baselineskip}%
\setlength{\parindent}{0pt}%

\newenvironment{answer}[1]{
\color{blue}
	{\bf Answer:}
}{
}

\newenvironment{alice}[1]{
\color{magenta}
	{\bf Alice:}
}{
}

\usepackage{mathtools}
\DeclarePairedDelimiter\ceil{\lceil}{\rceil}
\DeclarePairedDelimiter\floor{\lfloor}{\rfloor}

\title{CS 486/686 Project Group Contract Template}
\author{Shuheng Cao, Miaoqi(Frank) Zhang, Jiachen Zhang}
\date{\today}

\begin{document}

\maketitle

The purpose of the group contract is for the group members to create and communicate their expectations regarding how they will collaborate on the project.  If there any problem arises with a group, the course staff will use the group contract as a basis for their investigations.

To complete the group contract, please discuss the following questions as a group and fill out the answers below. (Feel free to delete Alice's comments in your submission.)

\begin{itemize}
\item What does each member of the group want to get out of working on this project? Is everyone here to accomplish the same thing? What are your goals as a group collectively?

\begin{answer}

Yes, learn new staff like reinforcement learning. Improve the group working skills and programming skills. Learn how to combine the things that we learn from lectures to actual project. By doing this project, we can also understand the course content in a deeper level.
\end{answer}

\item What group roles do you think are necessary for success of your project? Who will be assigned which group role? Take a look at a list of suggested roles on \href{https://uwaterloo.ca/student-success/sites/ca.student-success/files/uploads/files/TipSheet_GroupWork_0.pdf}{this handout}.  Consider each group member’s strengths and weaknesses, and how group roles can help everyone learn or capitalize on their strengths.

\begin{answer}

\begin{itemize}
  \item Human crafted engine.
  \item AI model. (training, inference, interacting)
  \item Connection between model and games.
  \item Visualization (including Tetris game implementation).
  \item Report writing.
\end{itemize}
Actually, we are not going to divide ourselves into specific roles. We are planning to be task-based. Each one of us will be assigned certain tasks after the meeting, and we will complete the task seperately, but also work together. So that everyone will have some understanding 
about different part of the work. This will also help us to learn more things and understand more about the contents. Shuheng Cao is more experienced in this topic, since he did some projects related to AI already, and he knows a bit more about this topic. So he acts more like a leader and figure out the direction that 
we will go. Miaoqi Zhang and Jiachen Zhang has some work experience about it, they will contribute to research and implementation. We will put all our efforts together and make full use of the strength of each one of us.
\end{answer}

\item How will you communicate? What are your expectations regarding the timeliness of responses to emails or other messages?  How often do you plan to meet?  What time will the meetings be and how long will the meetings be?  What technology will you use?  

\begin{alice}

I highly recommend that you schedule a weekly meeting at the beginning of the term and that you make it mandatory for every group member to attend the weekly meeting.  This meeting will help you keep on track with the project, identify problems and address the problems promptly.  

\end{alice}

\begin{answer}

We are going to communicate by using Zoom and WeChat. Since we are usually online and available via WeChat, so that the expectation of timeliness can easily achieve. 
We plan to have a meeting once a week, on each Friday afternoon. We schedule each meeting to be around 1 hour. We use zoom to hold the meeting.
\end{answer}

\item What do you expect group members to do prior to each weekly group meeting? 

\begin{answer}

We do not expect to finish the tasks together during the meeting, since it is very unefficient. We always assign tasks to each one of us at the end of the meeting every week. So we would expect everyone 
to finish their task before the next meeting, or if there's any problem or challenge, raise it as soon as possible and we will help to solve it together, and we can discuss during the meeting.
\end{answer}

\begin{alice}

I suggest that you produce a list of todo items at the end of every meeting.  Then, at the next meeting, you should go over the todo items from the previosu meeting and discuss how to proceed. 

\end{alice}


\item What are you expectations around the attitude of group members in the weekly meeting, and how you should respond to each other’s ideas?  How will you manage turn-taking? How will you ensure that all people contribute to the conversations? How would you ensure that decision making is thorough yet expedient?

\begin{answer}

We expect everyone to share their ideas during the meeting. Feel free to agree or disagree on an idea. Since we are all heading to the same purpose, all of us should contribute to the project. 
We are welcome to have different ideas during the meeting. Each one of us will explain our idea and point out the advantages and disadvantages. The other group members will think about whether the idea will work or not.
Then we will vote for the best approach. Since we already know each other pretty well, so we know that all of us will contribute to the conversations. We are all motivated to this project.
For the dicision making, since 3 people is not a large group. It will not be hard to agree on one decision. We would raise our ideas first, and then vote for a proper decision.
\end{answer}

\begin{alice}

You need to figure out a good way to structure your meetings so that you will get the most out of each meeting.  In addition, you need to figure out a way to make decisions when there is disagreement within the group.  For instance, do you always require consensus before making a decision or are you comfortable with making a decision based on majority vote?  
\end{alice}

\item How are you going to structure the work? Do you expect to do most of the work during or outside of the weekly group meeting? What process will you use to assign the responsibilities?

\begin{answer}

We will devide the work into a few approximately equal workload part, and assign the work to each one of us. We expect to do most of the work outside of the weekly group meeting. Group meeting is only for presenting our work from last week.
We will first discuss together to figure out how to break the work into parts. Then we will first volunteer to take the part that we may good at or interested in. If no one has specific passion for the part, then we will randomly assign work among us, since we have already agreed on the way to break into parts.
\end{answer}

\begin{alice}

It's fine to use a divide-and-conquer approach.  However, you should make sure to assign a similar amount of work to each group member such that no one feels overwhelmed or left out.  
\end{alice}

\item How will you submit the deliverables? Do you expect all members of the group to have a chance to vet the submission before submission? When should the write-up be ready for everyone to review?

\begin{answer}

We will first submit all of our works into one github repository so that all of us can take a look at our work and process. We do expect all members of the group to have a chance to vet the submission before submission. So that we can still have a chance to comment or improve on our work.
We are not going to submit everything in the last minute before the deadline. We expect everything will be ready one or two days before the deadline, so that we can do some adjustments if needed.
\end{answer}

\begin{alice}

It is a good idea to discuss your ``procrastination'' habit here.  (Let's be honest.  We all procrastinate to some extent. =)  You will want to set an early enough deadline such that every group member will have their work ready and there is enough time for other group members to go over the work and modify it appropriately.  Doing this can avoid situations such as ``my group member did not complete their work until the day when the report is due and they left no time for me to review and modify the report.''
\end{alice}

\item How will you deal with surprises? What should a group member do when they have a hard time delivering on something they promised? How will the group respond?

\begin{answer}

Do everything in advance. It is very common to have some unexpected challenges, and this is also the interesting part of the project. We could learn from the challenges and surprises. In order to prevent that the surprise will affect our project, we hope that we can do everything in advance, instead of doing everything in the last minute. 
If one of us is having a hard time, he should raise the problem as soon as possible. Not to wait until the meeting. We should keep in touch all the time. So that we can work together to solve the problem. We could have additional meetings if needed and discuss on the difficulty that we have. To help that person to overcome the challenge. We could also learn from the challenge.
\end{answer}

\begin{alice}

Dealing with unexpected situations is one of the most important skills that you will need to develop when working in a group.  Trust me --- with such an open-ended project, there will be surprises.  It is especially important to keep communicating when you are struggling.  The worst way of dealing with a difficult situation is to stop responding to emails and stop attending meetings.  Instead, let your group members know that you are struggling and try to figure out a solution together.
\end{alice}

\item How will you handle conflict? If any member in the group feels that something is not going right in the group, how would they signal it? How will the group respond?

\begin{answer}

If there are conflicts, we should discuss on time. Since we are a group of three and we are all nice people, if there's a conflict between two of us, the other one should stand up and help to organize everything in a fair and objective way. We should always think for others and try to understand others. As long as we are fighting for the same aim, we will always stay together.
If any memeber in the group feels that something is not going right in the group, he should point the problem out, in a objective way without being emotional, as soon as possible. Then he could also raise his ideas and suggestions. We will discuss on the problem and try to solve the problem together.
\end{answer}

\begin{alice}

Having conflicts between group members is inevitable.  It's important to establish some protocol regarding how you will resolve conflicts.  You are expected to try to resolve any conflict yourself first.  If you are unable to resolve the conflict, please notify the course staff and we will try to help you.

\end{alice}

\end{itemize}

\end{document}





























